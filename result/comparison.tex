\documentclass[a4paper,11pt,titlepage]{jsarticle}

\usepackage{amsmath}
\usepackage{amssymb}
\usepackage{amsfonts}
\usepackage{bm}
\usepackage[dvipdfmx]{graphicx}
\usepackage{listings}
% \usepackage{jvlisting}
% \usepackage{jlisting}
\usepackage{otf}
\usepackage{float}
\usepackage{url}
\usepackage{ascmac}
\usepackage{fancybx}
\usepackage{subcaption}
\usepackage{caption}
\usepackage[margin=1in]{geometry} % ページ余白を狭めて表示領域を広く
%===
\usepackage{multirow} % \multirowコマンドを使用するために必要
\usepackage{float}    % テーブルの配置オプション[H]を使用するために必要
\usepackage{caption}  % キャプションのカスタマイズや表のフロート環境を適切に扱うために推奨

% \bibliographystyle{junsrt} % スタイル(番号順)
% \bibliographystyle{plain}
% \bibliography{references}  % references.bib を指定(拡張子は不要)

\lstset{%ソースコードの表示に関する設定
    basicstyle={\ttfamily},
    identifierstyle={\small},
    commentstyle={\smallitshape},
    keywordstyle={\small\bfseries},
    ndkeywordstyle={\small},
    stringstyle={\small\ttfamily},
    frame={tb},
    breaklines=true,
    columns=[l]{fullflexible},
    numbers=left,
    xrightmargin=0zw,
    xleftmargin=3zw,
    numberstyle={\scriptsize},
    stepnumber=1,
    numbersep=1zw,
    lineskip=-0.5ex
}

% 画像ファイルを検索するディレクトリを指定
\graphicspath{ {images/} }
%これを設定しておけば,自動的にimages/から画像を参照してくれる。
%================================

\begin{document}


\begin{table}[htbp]
    \centering
    \caption{モデルごとの訓練精度および検証精度の比較}
    \label{tab:model_performance}
    \begin{tabular}{|l|l|c|c|c|}
        \hline
        \textbf{Model} & \textbf{} & \textbf{Epoch} & \textbf{Train Accuracy (\%)} & \textbf{Validation Accuracy (\%)} \\
        \hline
        \multirow{8}{*}{Custom Model} & \multirow{4}{*}{背景適用前} & 1 & 99.48 & 100 \\
        & & 2 & 100 & 99.99 \\
        & & \dots & \dots & \dots \\
        & & 10 & 100& 100 \\ % 背景適用前のエポック10を追加
        \cline{2-5}
        & \multirow{4}{*}{背景適用後} & 1 & 99.06 & 99.35 \\
        & & 2 & 99.52 & 99.5 \\
        & & \dots & \dots & \dots \\
        & & 10 & 99.68 & 99.58 \\ % 背景適用後のエポック10をグループ内に移動
        \hline
        \multirow{8}{*}{ResNet 18} & \multirow{4}{*}{背景適用前} & 1 & 97.92 & 99.28 \\
        & & 2 & 99.23 & 99.22 \\
        & & \dots & \dots & \dots \\
        & & 10 & 99.88 & 99.52 \\ % 背景適用前のエポック10を追加
        \cline{2-5}
        & \multirow{4}{*}{背景適用後} & 1 & 97.75 & 98.45 \\
        & & 2 & 99.24 & 99.39 \\
        & & \dots & \dots & \dots \\
        & & 10 & 99.59 & 99.39 \\ % 背景適用後のエポック10をグループ内に移動
        \hline
        \multirow{8}{*}{ResNet 34} & \multirow{4}{*}{背景適用前} & 1 & 97.74 & 99.08 \\
        & & 2 & 99.27 & 99.34 \\
        & & \dots & \dots & \dots \\
        & & 10 & 99.84 & 99.47 \\ % 背景適用前のエポック10を追加
        \cline{2-5}
        & \multirow{4}{*}{背景適用後} & 1 & 97.48 & 98.9 \\
        & & 2 & 99.13 & 99.25 \\
        & & \dots & \dots & \dots \\
        & & 10 & 99.56 & 99.45 \\ % 背景適用後のエポック10をグループ内に移動
        \hline
        \multirow{8}{*}{ResNet 50} & \multirow{4}{*}{背景適用前} & 1 & 95.26 & 98.58 \\
        & & 2 & 98.69 & 98.76 \\
        & & \dots & \dots & \dots \\
        & & 10 & 99.75 & 99.39 \\ % 背景適用前のエポック10を追加
        \cline{2-5}
        & \multirow{4}{*}{背景適用後} & 1 & 95.88 & 98.81 \\
        & & 2 & 98.85 & 97.76 \\
        & & \dots & \dots & \dots \\
        & & 10 & 99.54 & 98.94 \\ % 背景適用後のエポック10をグループ内に移動
        \hline
        \multirow{8}{*}{EfficientNetB0} & \multirow{4}{*}{背景適用前} & 1 & 94.47 & 98.76 \\
        & & 2 & 99 & 99.19 \\
        & & \dots & \dots & \dots \\
        & & 10 & 99.95 & 99.96 \\ % 背景適用前のエポック10を追加
        \cline{2-5}
        & \multirow{4}{*}{背景適用後} & 1 & 95.42 & 98.99 \\
        & & 2 & 98.98 & 99.13 \\
        & & \dots & \dots & \dots \\
        & & 10 & 99.57 & 99.54 \\ % 背景適用後のエポック10をグループ内に移動
        \hline
    \end{tabular}
\end{table}

\end{document}