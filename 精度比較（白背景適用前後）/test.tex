\documentclass[a4paper,11pt,titlepage]{jsarticle}

\usepackage{amsmath}
\usepackage{amssymb}
\usepackage{amsfonts}
\usepackage{bm}
\usepackage[dvipdfmx]{graphicx}
\usepackage{listings}
% \usepackage{jvlisting}
% \usepackage{jlisting}
\usepackage{otf}
\usepackage{float}
\usepackage{url}
\usepackage{ascmac}
\usepackage{fancybx}
\usepackage{subcaption}
\usepackage{caption}
\usepackage[margin=1in]{geometry} % ページ余白を狭めて表示領域を広く
%===
\usepackage{multirow} % \multirowコマンドを使用するために必要
\usepackage{float}    % テーブルの配置オプション[H]を使用するために必要
\usepackage{caption}  % キャプションのカスタマイズや表のフロート環境を適切に扱うために推奨

% \bibliographystyle{junsrt} % スタイル(番号順)
% \bibliographystyle{plain}
% \bibliography{references}  % references.bib を指定(拡張子は不要)

\lstset{%ソースコードの表示に関する設定
    basicstyle={\ttfamily},
    identifierstyle={\small},
    commentstyle={\smallitshape},
    keywordstyle={\small\bfseries},
    ndkeywordstyle={\small},
    stringstyle={\small\ttfamily},
    frame={tb},
    breaklines=true,
    columns=[l]{fullflexible},
    numbers=left,
    xrightmargin=0zw,
    xleftmargin=3zw,
    numberstyle={\scriptsize},
    stepnumber=1,
    numbersep=1zw,
    lineskip=-0.5ex
}

% 画像ファイルを検索するディレクトリを指定
\graphicspath{ {images/} }
%これを設定しておけば,自動的にimages/から画像を参照してくれる。
%================================

\begin{document}


\section{数値全然違うやんけ!}
\begin{table}[H]
\centering
\caption{Custom Modelにおける白背景適用前後の精度比較}
\label{tab:Custom_compare}
\begin{tabular}{l|rr|rr}
\hline
Epoch & Train (Before) & Val (Before) & Train (After) & Val (After) \\
\hline
1  & 99.21 & 99.33 & 99.06 & 99.35 \\
2  & 99.45 & 99.44 & 99.52 & 99.50 \\
3  & 99.53 & 99.45 & 99.58 & 99.47 \\
4  & 99.56 & 99.47 & 99.61 & 99.53 \\
5  & 99.59 & 99.51 & 99.63 & 99.58 \\
6  & 99.60 & 99.53 & 99.63 & 99.59 \\
7  & 99.63 & 99.52 & 99.65 & 99.57 \\
8  & 99.64 & 99.54 & 99.66 & 99.58 \\
9  & 99.65 & 99.55 & 99.65 & 99.57 \\
10 & 99.66 & 99.57 & 99.68 & 99.58 \\
\hline
\end{tabular}
\end{table}


% ===efficient
\begin{table}[H]
\centering
\caption{EfficientNet\_b0における白背景適用前後の精度比較}
\label{tab:Efficientnetb0_compare}
\begin{tabular}{l|rr|rr}
\hline
Epoch & Train (Before) & Val (Before) & Train (After) & Val (After) \\
\hline
1  & 96.81 & 99.31 & 95.42 & 98.99 \\
2  & 98.99 & 99.37 & 98.98 & 99.13 \\
3  & 99.31 & 99.36 & 99.33 & 99.34 \\
4  & 99.36 & 99.39 & 99.44 & 99.45 \\
5  & 99.44 & 99.41 & 99.55 & 99.45 \\
6  & 99.48 & 99.45 & 99.52 & 99.47 \\
7  & 99.51 & 99.42 & 99.57 & 99.21 \\
8  & 99.55 & 99.48 & 99.54 & 99.52 \\
9  & 99.56 & 99.50 & 99.59 & 99.52 \\
10 & 99.59 & 99.54 & 99.57 & 99.54 \\
\hline
\end{tabular}
\end{table}

% ===res18
\begin{table}[H]
\centering
\caption{ResNet18における白背景適用前後の精度比較}
\label{tab:Resnet18_compare}
\begin{tabular}{l|rr|rr}
\hline
Epoch & Train (Before) & Val (Before) & Train (After) & Val (After) \\
\hline
1  & 96.96 & 99.13 & 97.75 & 98.45 \\
2  & 99.19 & 99.38 & 99.24 & 99.39 \\
3  & 99.26 & 99.34 & 99.32 & 99.20 \\
4  & 99.33 & 99.29 & 99.41 & 99.12 \\
5  & 99.39 & 99.31 & 99.50 & 99.43 \\
6  & 99.44 & 99.33 & 99.50 & 99.15 \\
7  & 99.46 & 99.35 & 99.54 & 98.93 \\
8  & 99.49 & 99.38 & 99.53 & 99.45 \\
9  & 99.53 & 99.41 & 99.57 & 99.49 \\
10 & 99.56 & 99.43 & 99.59 & 99.39 \\
\hline
\end{tabular}
\end{table}



% === res34
\begin{table}[H]
\centering
\caption{ResNet34における白背景適用前後の精度比較}
\label{tab:Resnet34_compare}
\begin{tabular}{l|rr|rr}
\hline
Epoch & Train (Before) & Val (Before) & Train (After) & Val (After) \\
\hline
1  & 96.83 & 99.11 & 97.48 & 98.90 \\
2  & 99.01 & 99.26 & 99.13 & 99.25 \\
3  & 99.17 & 99.30 & 99.26 & 99.29 \\
4  & 99.24 & 99.32 & 99.33 & 99.04 \\
5  & 99.31 & 99.34 & 99.41 & 99.25 \\
6  & 99.38 & 99.35 & 99.45 & 99.31 \\
7  & 99.43 & 99.36 & 99.52 & 99.39 \\
8  & 99.47 & 99.39 & 99.54 & 99.12 \\
9  & 99.51 & 99.41 & 99.54 & 99.43 \\
10 & 99.53 & 99.42 & 99.56 & 99.45 \\
\hline
\end{tabular}
\end{table}

% === res50
\begin{table}[H]
\centering
\caption{ResNet50における白背景適用前後の精度比較}
\label{tab:Resnet50_compare}
\begin{tabular}{l|rr|rr}
\hline
Epoch & Train (Before) & Val (Before) & Train (After) & Val (After) \\
\hline
1  & 96.48 & 99.02 & 95.88 & 98.81 \\
2  & 98.94 & 99.25 & 98.85 & 97.76 \\
3  & 99.14 & 99.30 & 99.13 & 99.31 \\
4  & 99.26 & 99.33 & 99.27 & 99.45 \\
5  & 99.35 & 99.34 & 99.36 & 99.39 \\
6  & 99.39 & 99.36 & 99.42 & 98.50 \\
7  & 99.41 & 99.36 & 99.39 & 99.39 \\
8  & 99.45 & 99.37 & 99.48 & 99.35 \\
9  & 99.49 & 99.39 & 99.51 & 99.38 \\
10 & 99.52 & 99.40 & 99.54 & 98.94 \\
\hline
\end{tabular}
\end{table}


\end{document}
