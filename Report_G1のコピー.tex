\documentclass[a4paper,11pt,titlepage]{jsarticle}

\usepackage{amsmath}
\usepackage{amssymb}
\usepackage{amsfonts}
\usepackage{bm}
\usepackage[dvipdfmx]{graphicx}
\usepackage{listings}
%\usepackage{jvlisting}
%\usepackage{jlisting}
\usepackage{otf}
\usepackage{float}
\usepackage{url}
\usepackage{ascmac}
\usepackage{fancybx}
%\bibliographystyle{junsrt} % スタイル(番号順)
%\bibliographystyle{plain}
%\bibliography{references}  % references.bib を指定(拡張子は不要)

\lstset{%ソースコードの表示に関する設定
  	basicstyle={\ttfamily},
  	identifierstyle={\small},
  	commentstyle={\smallitshape},
  	keywordstyle={\small\bfseries},
  	ndkeywordstyle={\small},
  	stringstyle={\small\ttfamily},
  	frame={tb},
  	breaklines=true,
  	columns=[l]{fullflexible},
  	numbers=left,
  	xrightmargin=0zw,
  	xleftmargin=3zw,
  	numberstyle={\scriptsize},
  	stepnumber=1,
  	numbersep=1zw,
  	lineskip=-0.5ex
}
%================================

\begin{document}

%\title{知能情報実験\rom{3}(データマイニング班)\\ 美男・美女を識別し分類、一般人との顔写真と比較し差異を明確にする}
\title{知能情報実験 \ajroman{3}(データマイニング班)\\顔画像に基づく美男美女の識別と一般人との比較による特徴抽出}
\author{ 215706D, 235221E, 235701B, 235732B}
\date{提出日:\today}
\maketitle

\tableofcontents
\clearpage

\begin{abstract}
% 文書の概要
\end{abstract} 


\section{はじめに}
\subsection{実験の目的と達成目標}
知能情報実験\ajroman{3}は,情報工学分野のより専門的な知識を理解・習得することを目的として,半年間でシステムの開発やデータ解析等に取り組む実施される.
その中の一つデータマイニング班においては機械学習外観ならびにその応用を通し,対象問題への理解、特徴量抽出等の前処理,バージョン管理やデバッグ・テスト等を含む仕様が定まっていない状況下における開発方法,コード解説や実験再現のためのドキュメント作成等の習得を目指す.


\subsection{テーマ「美男・美女を識別し分類、一般人との顔写真と比較し差異を明確にする」とは}
\(本グループでは顔画像のデータセットを活用し,世界の美男・美女(有名人)と称される顔写真と一般人の顔写真を識別・分類するモデルの構築を問題として設定した.


本テーマでは,機械学習モデルから有名人の顔に共通する審美的特徴(黄金比に基づく顔の比率や対称性等)を特定し,一般人の顔画像との差異を明確化する.具体的には,顔写真から顔の輪郭や鼻立ち,肌の色調や表情,目等をデータに合わせて前処理を行い,全体データに対してラベルをつけ,カテゴリ化.そして,ResNet34モデルを用いて学習およびテストを行う,
このテーマに取り組む意義として,美しさや魅力等を黄金比や対称性を定量化・可視化でき,美的評価が客観的に認識できる.また,ResNet34モデルが人間の美的感覚に近い判断をどの程度行えるかというデータを明らかにする.
\)
%顔の特徴は,一般に言われているものは存在するものの,それを定量的に分析したものはない。


\section{実験方法}
\subsection{実験目的}
\(世界で美男・美女(有名人)と呼ばれる人の顔写真と一般人の顔写真を分類するモデルを作り,美男・美女と一般人の分類を行う.\)

\subsection{データセット構築}
\begin{itemize}
    \item 一般人のデータセット:FairFaceのデータセットを使用\\ 
    \item 美男・美女のデータセット:独自構築\\
            (1) 美男・美女の基準を決定\\
						\cite{bidanshi}や\cite{bijoshi}よりtop50までを2024年度の美男・美女のランキングとして扱う\\
            (2) データの収集\\
            web scrapingを利用し,webブラウザのbingでデータを収集する.\\
						\begin{itemize}
						\item ソースコードを以下のGitHubにて公開する。
            \item \url{https://github.com/e235221/info3dm\_racial\_classification}
            \end{itemize}
            (3) 前処理\\
            収集したデータの顔部分だけ切り取り,サイズを$300 \times 300$とし,FairFaceのデータセットと合わせて調整する.\\
            (4)ラベル付け\\
            収集したデータにFairFaceのデータセットと同様の形でラベル付きを行う.
    \item アップサンプリング・ダウンサンプリング\\
            (1) \cite{hopenet}を利用し,データセット(一般人+美男・美女)の中で正面と側面を分類\\
            (2) 側面のデータを除外\\
            (3) 各画像に対して左右反転を行い,美男・美女のデータセットにアップサンプリングを適用する。\\
            (4) 美男・美女のデータセットのうちランダムに選択することで一般人のデータセットをダウンサンプリングを適用する。
    \item 全体データセットにラベル付きを実行(美男・美女/一般人)

\end{itemize}

\subsection{モデル選定}
ResNet34を選択した。これはFairFaceで提供するデータセットの学習済みモデルをカスタムすることで実験を行なっているためである。詳しくは\cite{karkkainenfairface}を参照されたい。


\section{予定していた実験計画}
fairfaceに追加学習をすることによって有名人と一般人の差異を明らかにすることであった。しかし,精度が100\% になった。そこで,他のアプローチとしてresnet18,34,50とefficient netで新規に一般人と有名人の学習を行い,その結果を見る。



\subsection{パラメータ調整}




\section{実験結果}
\(\)




\section{考察}
\(\)




\section{意図していた実験計画との違い}
\(\)




\section{まとめ}
\(\)

\begin{thebibliography}{9}
  \bibitem{karkkainenfairface}
  Karkkainen, Kimmo and Joo, Jungseock.
  FairFace: Face Attribute Dataset for Balanced Race, Gender, and Age for Bias Measurement and Mitigation.
  In \textit{Proceedings of the IEEE/CVF Winter Conference on Applications of Computer Vision}, pages 1548--1558, 2021.
	\bibitem{bidanshi}
	Most Handsome Man In The World 2024,shiningawards.com, \url{https://shiningawards.com/most-handsome-man-in-the-world-2024/}
	\bibitem{bijoshi}
	Most Beautiful Faces 2024, gigazine.net, \url{https://gigazine.net/gsc_news/en/20241229-most-beautiful-faces-2024/}
	\bibitem{hopenet}
	Head Pose Estimation, \url{https://github.com/natanielruiz/deep-head-pose}
\end{thebibliography}


\end{document}
